\documentclass[20pt]{article}
\usepackage[utf8]{inputenc}
\usepackage{amsmath, amssymb, amsthm}
\usepackage{titlesec}
\usepackage{chngcntr}
\usepackage{tikz}
\usepackage{pgfplots}
% Customization -------

% Paper size, margin
\usepackage[letterpaper,top=1.5cm,bottom=1.5cm,left=1.75cm,right=1.75cm,heightrounded]{geometry}

% line height
\renewcommand{\baselinestretch}{1.15} % line space

% Paragraph indentation 
\setlength{\parindent}{0pt} % no indent

% Paragraph spacing
\setlength{\parskip}{0.8em} % space between paragraphs

% Section number formatting
\titleformat{\section}[hang]{\bfseries}{Problem \thesection\ }{0pt}{}

% --------------------

\title{Homework 02}
\author{Farbod Mohammadzadeh\\1008360462}
\date{11 October 2023}

\begin{document}
\Large
\maketitle

\newpage

\section*{Problem 2.1}

Consider the list of $n \ n-vectors$

\begin{equation}
    a_1 = \begin{bmatrix}
        1      \\
        0      \\
        0      \\
        \vdots \\
        0
    \end{bmatrix}, a_2 = \begin{bmatrix}
        1      \\
        1      \\
        0      \\
        \vdots \\
        0
    \end{bmatrix}, \dots, a_n = \begin{bmatrix}
        1      \\
        1      \\
        1      \\
        \vdots \\
        1
    \end{bmatrix}
\end{equation}

(The vector $a_i$ has its first $i$ entries equal to one, and the remaining entries zero.) Describe
what happens when you run the Gram–Schmidt algorithm on this list of vectors, $i.e.$, say
what $q_1, \dots , q_n$ are. Is $a_1, \dots , a_n$ a basis?

\underline{Solution:}

If we apply the Gram-Schmidt algorithm on the given list of vectors, we will through each iteration remove the linearly dependent elements of each $a_i$ vector until we arrive at the following result:

\begin{equation}
    q_1 = \begin{bmatrix}
        1      \\
        0      \\
        0      \\
        \vdots \\
        0
    \end{bmatrix}, q_2 = \begin{bmatrix}
        0      \\
        1      \\
        0      \\
        \vdots \\
        0
    \end{bmatrix}, \dots, q_n = \begin{bmatrix}
        0      \\
        0      \\
        0      \\
        \vdots \\
        1
    \end{bmatrix}
\end{equation}

Since the set of $a_1, \dots , a_n$ is entirely linearly independent it is a basis for $\mathbb{R}^n$.
\newpage
\section*{Problem 2.2}
\newpage
\section*{Problem 2.3}
\newpage

\section*{Problem 2.4}
\newpage

\section*{Problem 2.5}

\end{document}