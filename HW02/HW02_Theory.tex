\documentclass[20pt]{article}
\usepackage[utf8]{inputenc}
\usepackage{amsmath, amssymb, amsthm}
\usepackage{titlesec}
\usepackage{chngcntr}
\usepackage{tikz}
\usepackage{pgfplots}
% Customization -------

% Paper size, margin
\usepackage[letterpaper,top=1.5cm,bottom=1.5cm,left=1.75cm,right=1.75cm,heightrounded]{geometry}

% line height
\renewcommand{\baselinestretch}{1.15} % line space

% Paragraph indentation 
\setlength{\parindent}{0pt} % no indent

% Paragraph spacing
\setlength{\parskip}{0.8em} % space between paragraphs

% Section number formatting
\titleformat{\section}[hang]{\bfseries}{Problem \thesection\ }{0pt}{}

% --------------------

\title{Homework 02}
\author{Farbod Mohammadzadeh\\1008360462}
\date{04 October 2023}

\begin{document}
\Large
\maketitle

\newpage

\section*{Problem 2.1 (Gram-Schmidt algorithm)}

In this problem we use the notation $ProjS (x)$ to denote the projection of a vector $x$ onto some set
$S$, which consists of vectors that are of same dimension as $x$. Consider the following vectors and
subspaces.

In this problem we use the notation $ProjS (x)$ to denote the projection of a vector $x$ onto some set
$S$, which consists of vectors that are of same dimension as $x$. Consider the following vectors and
subspaces.

\end{document}